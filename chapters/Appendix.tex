\chapter*{Appendix I}
\addcontentsline{toc}{chapter}{Appendix I}
\label{chapter:appendix1}

\textbf{Ulceration risk of pressure-time relation}

\begin{equation}
    P(T) = \frac{P_{max}-P_{min}}{1 + e^{\lambda(T-T_0)}} + P_{min}
\end{equation}

Here,

\begin{align*}
    P_{max} &= 31 kPa\\
    P_{min} &= 8 kPa\\
    \lambda &= 0.15 min^{-1}\\
    T_0     &= 95 min
\end{align*}


\begin{equation}
    T_{max}(P) = 
    \begin{cases}
        \infty & P < P_{min}\\
        0  & P > P_{max}\\
        T_0 + \frac{1}{\lambda}\left( \frac{P_{max}-P_{min}}{P-P_{min}} - 1 \right) & P_{min} < P < P_{max}
    \end{cases}
\end{equation}

\begin{equation}
    \delta R = \frac{\delta t}{T_{max}(P)}
\end{equation}


\chapter*{Appendix II}
\addcontentsline{toc}{chapter}{Appendix II}
\label{chapter:appendix2}

Schematic of the pressure mat
\begin{figure*}[h]
      \vspace{-0.3cm}
      \centering
      \includegraphics[width=0.8\textheight, angle=90]{figs/schematic1.pdf}
      \caption[Schematic]{Schematic Diagram.}
      \label{fig:schematic}
      \vspace{-1cm}
\end{figure*}


% \textbf{Mean Field Algorithm} \\
% \input{figs/algo.tex}


\chapter*{Appendix III}
\addcontentsline{toc}{chapter}{Appendix III}
\label{chapter:appendix3}

ATMega code

\cite{Anurag17}