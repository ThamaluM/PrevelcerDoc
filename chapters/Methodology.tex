\chapter{Methodology}
\label{chapter:method}

An information system architecture that supports care planning of pressure ulcers requires certain basic functionalities such as 
\begin{itemize}
	\item capturing bio-mechanical data of the body
	\item Analyzing those data
	\item collecting risk assessment data
	\item providing platform to report and document ulcers
	\item networking related people
	\item planning schedules
\end{itemize}

Therefore our information system architecture is supported by pressure sensing mats and mobile app. Some standard risk assessment scales, ulcer documentation formats are added to the system with appropriate modifications. Additionally scalability, flexibility and cost-effectiveness are two other important characteristics of such system. Our initial scope was to build a pressure sensing mattress system that is capable of recommending optimal repositioning strategies based on bio-mechanical data. As there are no proper evaluation criteria to assess pressure ulcer  prevention and the existing bio-mechanical and pathological research in pressure ulcers are inconclusive, we constructed an information system that provide a platform to investigate pressure ulceration phenomenon while providing a tool for care planning by digitizing processes currently done in paper or not done in any systematic way. Existing theories can be used in our system to improve care planning within their limitation.

\section{Component Diagram}




\section{Information system back-end}


Information system backend is written in python using the enterprise level web fullstack designing framework Django and hosted in Heroku cloud platform. Postegresql an enterprise level SQL based relational database management tool is used as the database management system. All the static media files are stored in a Cloudinary S3 bucket. APIs are created from django-rest-framework platform and firebase is used to communicate with mobile apps as push notifications. 

\subsection{Authentication and Authorization}

There are user accounts to authenticate the users and there are three groups as doctors, caretakers and patients. These roles and accounts are used to authorize access to particular components. Only users have write or update permission to their personal information, care takers can update there risk assessment data while doctors can update ulcer reporting documentation as well as risk assessment data. Even latter data only accessible to caretakers or doctors who are assigned to relevant patients.


\subsection{Social Networking}

All doctors, caretakers, patients can be see each other in search lists. The connection between the users are established via request and confirm mechanism. There are request, show, accept, reject, delete functionalities for a request.

\subsection{Pressure data}

Pressure data sent from pressure mats are stored in the database via the server. These data are further analysed with Neural Network Models to find ulceration points. Pressure data is stored in the format 

lx, ly, x, y, p, n format. This format allows us to send cell readings one by one and flexible the system for any resolution. This pressure data is used to create the mat.

\subsection{Machine Learning}

There are two machine learning models to analyze pressure data. One is to identify pressure and the other is to identify ulceration points

Neural network training we used a dataset by university of Dallas and we used data preprocessing and augmentations.


\subsection{Scheduling}

There is a specific risk scale by university of Dallas and using this risk scale we could find that the order is right left shifting and half of the time for supine.

\subsection{Risk Assessment}

Braden scale and additions to the risk assessment scales

\subsection{Ulcer documentation}

Ulcer documentation is based on SOS guidelines and appropriate modifications.




\section{Mobile app} 

Mobile app is developed by using React Native which is a cross platform framework (Android and IOS development framework.)

Push notifications are send to the app.




\section{Pressure Mat}


The selected material was Velostat. 

\subsection{Individual sensor calibration}

 Material testing. Velostat was selected. We obtained curves for Velostat.


\subsection{Preparing Mat}

Multiplexers and ATMega code was written. Node MCU was used for communications. UART and cross talk handling.




