\chapter*{Abstract}
\addcontentsline{toc}{chapter}{Abstract}

\begin{center}
	\vspace{5mm}
	\MakeUppercase{\textbf{Health Information System with Pressure Sensing towards A Better Care Plan for Pressure Ulcers}}\\
	\vspace{5mm}
	Group Members: \memberA, \memberB, \memberC, \\ \memberD \\
	\vspace{5mm}
	Supervisors: \supervisorA %, \supervisorB \\
	\vspace{5mm}
\end{center}

\noindent Keywords: Pressure Ulcer, Bedsore, Reposition, Health Information System, Pressure sensing\\


Pressure ulcers (bedsores) are a major problem in health care settings as it causes severe complications. Therefore prevention at the early stage is more beneficial over treatments. Pathophysiological knowledge about pressure ulceration is incomplete as there are insufficient evidence supporting existing theories. Bed ridden patients who cannot change positions on their own are prone to pressure ulcers. The efficacy of prevention methods like frequent patient repositioning and pressure redistributing support surfaces are challenged by recent systematic reviews against the higher face validity. This project is to develop a health information system which collects data related to pressure distributions, personal risk assessment and documentation of pressure ulcer occurance. Patients in various settings and other relevent parties like doctors and caretakers can be connected together with the information system with mobile apps and pressure measuring mattresses. There is a mobile app and a low cost pressure measuring mat. Pressure measuring mat is a low cost solution using the piezoresistive material Velostat. Pressure images from the mats are used identify postures and locations of the ulceration sites in order to record pressure at these ulceration points.
