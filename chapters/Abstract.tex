\chapter*{Abstract}
\addcontentsline{toc}{chapter}{Abstract}

\begin{center}
	\vspace{5mm}
	\MakeUppercase{\textbf{Health Information System with Pressure Sensing towards A Better Care Plan for Pressure Ulcers}}\\
	\vspace{5mm}
	Group Members: \memberA, \memberB, \memberC, \\ \memberD \\
	\vspace{5mm}
	Supervisors: \supervisorA %, \supervisorB \\
	\vspace{5mm}
\end{center}

\noindent Keywords: Pressure Ulcer, Bedsore, Reposition, Health Information System, Pressure sensing\\


Pressure ulcers (bedsores) remain a major problem in health care settings as it causes severe complications. Therefore prevention at the early stage is more beneficial than treatments. Pathophysiological knowledge about pressure ulceration is incomplete as there is insufficient evidence supporting existing theories. Bedridden patients who cannot change positions on their own are prone to pressure ulcers. The efficacy of prevention methods like frequent patient repositioning and pressure redistributing support surfaces are challenged by recent systematic reviews against the higher face validity. The scarcity of detailed data is an obstacle to further research. Collecting long term data from various patients in different settings is challenging. This project delivers a health information system that collects data about pressure distributions, patient risk assessment and documentation of pressure ulcer occurrences. Patients, caretakers and doctors can be connected to the information system with mobile apps and pressure measuring mats. A low-cost solution for pressure measuring mat is introduced using the piezoresistive material Velostat. Thirty-two rows of copper strips are pasted on one of the neoprene sheets and sixteen columns on another. Then the Velostat sheet is sandwitched in between the two neoprene sheets. Perpendicularly intersecting copper strips work as a single cell. Altogether, 512 cells are produced. Pressure images from the mat are used to identify postures and locate ulceration points to obtain the pressure at these sites. We can control the cells and measure pressure with analogue multiplexers as the Velostat and two neoprene sheets operate as a sensor matrix. Measured pressure values are communicated to the information system through the internet. Two neural network models are used to identify the sleeping postures and segment ulceration points. The pressure at each ulceration point is stored separately. The information system facilitates personal risk assessment and ulcer documentation. The risk assessment scale is a development of existing systematic scales. Ulcer documentation is a modification of several existing guidelines. The mobile app works as a user interface and sends push notifications to caretakers informing patient reposition schedules. The complete system serves different functionalities to different parties. It will facilitate researchers as a database with descriptive data, doctors as a reporting system of patients' details and history, caretakers as a tool to semi-automate reposition plan and guardians of the patients as a tool to supervise the caretakers. 