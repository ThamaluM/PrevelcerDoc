\chapter*{Abstract}
\addcontentsline{toc}{chapter}{Abstract}

\begin{center}
	\vspace{5mm}
	\MakeUppercase{\textbf{Health Information System with Pressure Sensing towards A Better Care Plan for Pressure Ulcers}}\\
	\vspace{5mm}
	Group Members: \memberA, \memberB, \memberC, \\ \memberD \\
	\vspace{5mm}
	Supervisors: \supervisorA %, \supervisorB \\
	\vspace{5mm}
\end{center}

\noindent Keywords: Pressure Ulcer, Bedsore, Reposition, Health Information System, Pressure sensing\\


Pressure ulcers (bedsores) are a major problem in health care settings as it causes severe complications. Therefore prevention at the early stage is more beneficial over treatments. Pathophysiological knowledge about pressure ulceration is incomplete as there are insufficient evidence supporting existing theories. Bed ridden patients who cannot change positions on their own are prone to pressure ulcers. The efficacy of prevention methods like frequent patient repositioning and pressure redistributing support surfaces are challenged by recent systematic reviews against the higher face validity. Scarcity of systematic data is an obstacle over further research. Collecting long term data from various patients in various settings is challenging.This project is to develop a health information system which collects data related to pressure distributions, patient risk assessment and documentation of pressure ulcer occurrences. Patients in various settings and other relevant parties like doctors and caretakers can be connected together with the information system with mobile apps and pressure measuring mattresses. Pressure measuring mat is a low cost solution using the piezoresistive material Velostat. Pressure images from the mats are used to identify postures and locations of the ulceration sites in order to record pressure at these ulceration points. Velostat sheet was sandwiched between two neoprene sheets containing columns or rows of copper tapes. This works as a sensor matrix and we can control the cells and measure pressure with analog multiplexers. Measured pressure values are communicated to the information system through internet. Two neural network models are used to identify the sleeping postures and segment ulceration points. The pressure at each ulceration point is stored separately. The information system provides personal risk assessment and ulcer documentation. Risk assessment scale is based on existing systematic scales. Ulcer documentation is developed using several existing guidelines and several new modifications. Mobile apps work as a user interface and support sending push notifications to caretakers informing patient reposition schedules. The complete system serves different functionalities to different parties. It will support researchers as a database with descriptive data, doctors as a reporting system of patients' details and history, caretakers as a tool to semi-automate reposition plan and guardians of the patients as a tool to supervise the caretakers. 