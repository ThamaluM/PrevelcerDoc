\chapter{Introduction}

Pressure ulcers are a major problem in health care due to high prevalence and high cost of treatment. Some ulcers do not heal for decades. If not properly managed, pressure ulcers may cause complications such as septicemia or even death. 

\section{Literature Review}
Prolonged external pressure into bony areas of body causes pressure ulcers in bed-ridden patients. Prevailing patho-physiological understanding of pressure ulceration is very incomplete. Existing theories suggest that reduction of oxygen supply to skin tissues causes cell death through an ischaemia-reperfusion cycle, which results in pressure ulcer formation. Another theory suggests that internal pressure on muscle tissues by bones causes ulceration. None of these theories are empirically verified. Reswick and Rogers studied effect of pressure and time on cell death in 1976. There is no considerable improvement since then other than a few papers suggesting slight modifications. There is no satisfactory empirical research on internal pressure theory, although Deep Tissue Injury (DTI), a recently defined category of pressure ulcers, is widely believed to be a result of internal pressure from bones.

Exact bio-mechanical impact of pressure on human body (skin and muscles) is not yet known given that only a few unsatisfactory bio-mechanical models from early research based on animal testing and qualitative speculations without quantitative data are available. There is no sufficient data on the mechanism through which external pressure causes an ischemia-reperfusion cycle. Furthermore, there is no quantitative empirical evidence for the effect of other proposed bio-mechanical factors such as shear, friction and moisture also. 

The main pressure ulcer prevention strategy which is strongly recommended by health care authorities is frequent patient repositioning. This strategy was popularized after the end of World War II, based on face validity. Caretakers should turn the patients into a different sleeping posture for every 2 h.
The absence of high quality research evidence supporting the efficacy of repositioning was discussed in a Cochrane systematic review published in 2014 (and updated in 2020). The research does not show significant advantage of 2 h repositioning over alternative time periods or no-repositioning. Currently available data are low certain and not sufficiently reliable to provide a conclusion. Existing studies do not consider bio-mechanical facts explicitly. Recently some researchers castigated the repositioning strategy for side effects such as disturbance to sleeping patterns, negative impact on dementia patients and back pain of caretakers. Recently NICE guidelines increased the time period from 2 h to 6 h for normal patients and 4 h for patients in high-risk category. The efficacy of other proposed prevention strategies including the use of pressure redistribution surfaces also are not supported with research, contrary to the availability of wide range of products in the market.

There are several patient risk assessment indicators including Braden and Waterlow scales. The importance of a systematic scale for risk assessment is often emphasized over using clinical judgement alone. Clinical evidence on the efficacy of these tools is still insufficient and uncertain.
Proper documentation is of crucial importance in modern health care. There are several paper-based or electronic documentation systems for pressure ulcers. According to studies, the purpose of existing documentation systems is not met with ulcer prevention and care. The patient repositioning plans are rarely documented. There are no records of bio-mechanical data. Existing electronic documentation systems are desktop applications that store records inside a single end-user device. The recent advancement of mobile, web, IOT and cloud technologies are not yet employed for pressure ulcer documentation. 

Ajami and Khalegi discussed the importance of a wireless sensor network for pressure monitoring. There is need for a sophisticated Health Information System (HIS) that supports not only remote pressure monitoring and electronic documentation but also important utilities to optimize care plan such as posture detection, ulceration point (the specific areas of body which are more prone to ulceration) detection, pressure/risk estimation, repositioning schedule calculation and carer notification. Although there are several implementations addressing subsets of above tasks already, constructing a health information system in a holistic point of view is a novel concern. Such information system should network patients, caretakers, guardians and doctors together and generate, collect, store, analyse and interpret data for better care planning. Considering the fact that pressure ulcer is a grey area of medical research the information system should work as tool to investigate bio-mechanical and clinical details of bed-ridden patients. Another requirement is to provide support for semi-automation of patient care through care plan optimization like reposition schedule calculation and carer notification system. Wide availability of mobile technology paves a way for a flexible and more sophisticated reporting system. Integrating low cost pressure sensing equipment to the information system provides opportunity to monitor and collect data for long periods and to widen the scope of research including the majority of ulcer-prone patients that reside in household settings. Reduction of cost will facilitate research into pressure ulcers in developing countries.
