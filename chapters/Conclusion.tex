\chapter{Discussion and Conclusion}

Pressure ulceration is yet an incompletely explored territory. Previous engineering research on pressure ulceration tried to make strong conclusions ignoring scarcity of evidence. As the existing prevention strategies including reposition were strictly challenged since 2014, it is unwise to make bold moves. Therefore we primarily focused on providing a platform for researchers to explore the phenomena while using available knowledge for better care plans. In addition, low-cost pressure measuring mat was introduced to promote research in developing countries. 

Although there were several applications for digitized ulcer documentation or pressure monitoring, they are standalone software that was expected to be run on a single machine. We introduce a cloud-based central network that will be helpful to collect data in one place. Previous engineering applications did not introduce mobile phones into ulceration care planning. The wide popularity of smartphones has transformed the nature of the problem. 

When the bounding boxes are marked, two inputs (pressure image and the label of the particular ulceration point is) given to find relevant bounding box parameters. It is considerably different from usual image segmentation models that predicts all the bounding boxes at once or classify bounding boxes. (Similar trick is commonly used with transformers) 

The terminology in ulcer documentation is localized to Sri Lanka and some new items were added as an improvement.

As all parts of the system are loosely coupled, any part can be integrated or used for other related applications.

As future work, textile production techniques can be applied to the pressure mat to build a commercial product. Live streaming of pressure images can be added to the app. If a sufficient amount of pressure and ulcer documentation data are available, researchers can investigate the exact relationship between pressure and pressure ulceration to find a better prevention strategy.


