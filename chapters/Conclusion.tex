\chapter{Discussion and Conclusion}

The pressure ulceration is yet an incompletely explored territory. Previous engineering research on pressure ulceration tried to make strong conclusions ignoring scarcity of evidence. As the existing prevention strategies including reposition were strictly challenged since 2014, it is unwise to make bold moves. Therefore we primarily focused on providing a platform for researchers to explore the phenomena while using available knowledge for better care plans. In addition low cost pressure measuring mat was introduced to promote research in developing countries. 

Although there were several applications for digitized ulcer documentation or pressure monitoring, they are standalone software that was expected to be run on a single machine. We introduce a cloud based central network that will be helpful to collect data into one place. Previous engineering applications did not introduce mobile phones into ulceration care planning. Wide popularity of smart phones has transformed the nature of the problem. 

When the bounded boxes are marked, two inputs (pressure image and the label of the particular ulceration point is) given to find relevant bounding box parameters. This is considerably different from usual image segmentation models which predicts all the bounding boxes at once or classify bounding boxes into categories. (Similar trick is usually used with transformers) 

The terminology in ulcer documentation is localized to Sri Lanka and several important factors were newly added.

As all the components of the system are loosely coupled parts of the system can be integrated or used for other related applications.

As future work proper textile production techniques can be applied to the mat to build a commercial mat. Live streaming of pressure image can be add to the app. If sufficient amount of pressure and ulcer documentation data are available researchers can investigate the exact relationship between pressure and pressure ulceration in order to find a better prevention strategy.


